%!TEX root=../main.tex

\chapter{How to use this template}
This chapter provides guidelines on how to effectively use this \texttt{LaTeX} template for your documents. It covers adding new chapters, inserting code samples, citing references, including graphics, and using automated cross-references.

\newpage

\section{Adding New Chapters}
To add a new chapter to your document, create a new \texttt{LaTeX} file in the chapters directory and include it in the main document using the \mintinline{latex}|\input{chapters/your-chapter.tex}| command. Here is an example of a new chapter file:

\begin{minted}{TeX}
\chapter{Chapter Title}
Explanation of the chapter content goes here.

\newpage

\section{Section Title}
You can add sections within the chapter as well.
\end{minted}

\section{Adding Code Samples}
To include code samples in your document, use the `minted` environment. This environment provides robust syntax highlighting and is especially useful for displaying source code. Here is an example:

\begin{verbatim}
\begin{minted}{c}
    #include <stdio.h>
    int main() {
        printf("Hello, World!");
        return 0;
    }
\end{minted}
\end{verbatim}

\textbf{Note:} If you are using VS Code, you can make use of a snippet to quickly insert these environments by typing \texttt{code} or \texttt{snippet}.

\section{Citing References and Direct Quotes}
Direct quotes in your text must be enclosed using \textbackslash enquote{}. Citations should be placed immediately after the punctuation that ends the sentence containing the quoted material. Here is how to properly cite and quote:
\begin{minted}{TeX}
This is a referenced sentence.\cite{RefLabel}
\enquote{This is a direct quote.}\cite{RefLabel}
\end{minted}

\section{Adding References}
Your references should be stored in the \texttt{references.bib} file. Each reference should have a unique label that you can use to cite it in your document. Here is an example of a reference entry:
\begin{minted}{bibtex}
@Online{RefLabel,
    author = {Author Name},
    title = {Title of the Article},
    year = {2021},
    url = {https://www.example.com},
    urldate = {2021-01-01}
}
\end{minted}

It is recommended to use a reference manager like JabRef to manage your references.

\section{Including Graphics}
To include graphics, use the \textbackslash includegraphics command from the `graphicx` package and wrap it inside a figure environment. Example:
\begin{minted}{TeX}
\begin{figure}[htp]
    \centering
    \includegraphics[width=0.5\textwidth]{assets/your-image.png}
    \caption{This is a caption.}
    \label{fig:example}
\end{figure}
\end{minted}

\textbf{Note:} If you are using VS Code, you can make use of a snippet to quickly insert these environments by typing \texttt{img}.


\subsection{Referencing Figures and Listings}
Automatically reference figures, tables, or code listings using \textbackslash autoref. Label your items and reference them as shown:
\begin{minted}{TeX}
As shown in \autoref{fig:example}, this figure demonstrates...
\end{minted}

\section{Glossary Entries and Acronyms}

To add a glossary entry, use the \texttt{\textbackslash newglossaryentry} command in the \texttt{glossary.tex} file. Here is an example:

\begin{minted}{TeX}
\newglossaryentry{api}{
    name={API},
    first={Application Programming Interface (API)},
    plural={APIs},
    description={Application Programming Interface}
}
\end{minted}

To reference the glossary entry in your document, use the \texttt{\textbackslash gls} command. For example:

\begin{minted}{TeX}
The \gls{api} provides a set of functions...
\end{minted}

You can also use the \texttt{\textbackslash glspl} command to make the first use of the term plural. For example:

\begin{minted}{TeX}
\glspl{api} are used to...
\end{minted}

\section{Noting To-Do Items}

To mark a to-do item in your document, use the \texttt{\textbackslash todo} command. For example:

\begin{minted}{TeX}
\todo{Add more details here}
\end{minted}

This will also throw a warning in the document, making it easier to spot and address later.
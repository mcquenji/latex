% Template for LaTeX projects to streamline setting up new projects.
% This template provides a basic structure for a document, including common packages and configurations.

\documentclass[11pt, a4paper]{report} % Specifies the document class and options

%% Packages
% Load packages to extend the functionality of LaTeX.

\usepackage[utf8]{inputenc} % Allows input encoding of international characters.
\usepackage[T1]{fontenc}    % Allows the use of different font encodings.
\usepackage{amsmath}        % Enhances math typesetting.
\usepackage{graphicx}       % Provides interfaces for including graphics.
\usepackage[colorlinks,urlcolor=blue,linkcolor=blue,citecolor=blue]{hyperref}       % Provides hyperlinking abilities in PDF.
\usepackage[english]{babel} % Sets the document language to English. Change to 'ngerman' for German.
\usepackage{csquotes}       % Provides advanced facilities for inline and display quotations.
\usepackage{parskip}        % Adds spacing between paragraphs.
\usepackage{fancyhdr}       % Customizes headers and footers.
\usepackage{bookmark}
\usepackage{titling}
\usepackage{lastpage}       % Allows referencing the last page of the document.
\usepackage{minted}         % For syntax highlighting code
\usepackage[backend=bibtex, style=ieee]{biblatex}
\usepackage{soul}           % for highlighting text
% Add more packages here as needed.

%% Page Layout
% Adjust the page dimensions and margins.

\usepackage{geometry}
\geometry{
  top=2.5cm,
  bottom=2.5cm,
  left=2.5cm,
  right=2.5cm
}

% Make citations superscript as seen in wikipedia articles.
\let\origcite\cite
\renewcommand{\cite}[1]{\textsuperscript{\origcite{#1}}} 

% Make quotes italicized.
\let\oldenquote\enquote
\let\endoldquote\endquote
\renewcommand{\enquote}[1]{\textit{\oldenquote{#1}}}

% Create a todo command to highlight unfinished sections.
\newcommand{\todo}[1]{\textcolor{red}{\hl{\textbf{TODO:} #1}}\PackageWarning{TODO:}{#1!}} 

% Remove indentation from paragraphs.
\setlength{\parindent}{0cm}

%% Bibliography
\bibliography{references} % References are managed in the references.bib file.

%% Title and Author
% Define the title, author, and date for the document.

\title{Document Title} % The title
\author{Author Name}  % The author
\date{\today}         % Sets date to current date.

%% Header and Footer
\pagestyle{fancy}

\fancyhead[L]{\nouppercase{\leftmark}}
\fancyhead[C]{}
\fancyhead[R]{\thetitle}

\fancyfoot[L]{\theauthor}
\fancyfoot[C]{}
\fancyfoot[R]{\thepage/\pageref{LastPage}}

\renewcommand{\headrulewidth}{0.4pt} % Adds a header rule line.
\renewcommand{\footrulewidth}{0.4pt} % Adds a footer rule line.

%% Document
\begin{document}

\maketitle  % Generates the title.

\newpage

\tableofcontents % Generates a table of contents based on the document's structure.

\newpage

%!TEX root=../main.tex

\chapter{\iflanguage{ngerman}{Kurzfassung}{Abstract}}
This is a simple abstract to describe your document's overview.

%!TEX root=../main.tex

\chapter{How to use this template}
This chapter provides guidelines on how to effectively use this \texttt{LaTeX} template for your documents. It covers adding new chapters, inserting code samples, citing references, including graphics, and using automated cross-references.

\newpage

\section{Adding New Chapters}
To add a new chapter to your document, create a new \texttt{LaTeX} file in the chapters directory and include it in the main document using the \mintinline{latex}|\input{chapters/your-chapter.tex}| command. Here is an example of a new chapter file:

\begin{minted}{TeX}
\chapter{Chapter Title}
Explanation of the chapter content goes here.

\newpage

\section{Section Title}
You can add sections within the chapter as well.
\end{minted}

\section{Adding Code Samples}
To include code samples in your document, use the `minted` environment. This environment provides robust syntax highlighting and is especially useful for displaying source code. Here is an example:

\begin{verbatim}
\begin{minted}{c}
    #include <stdio.h>
    int main() {
        printf("Hello, World!");
        return 0;
    }
\end{minted}
\end{verbatim}

\textbf{Note:} If you are using VS Code, you can make use of a snippet to quickly insert these environments by typing \texttt{code} or \texttt{snippet}.

\section{Citing References and Direct Quotes}
Direct quotes in your text must be enclosed using \textbackslash enquote{}. Citations should be placed immediately after the punctuation that ends the sentence containing the quoted material. Here is how to properly cite and quote:
\begin{minted}{TeX}
This is a referenced sentence.\cite{RefLabel}
\enquote{This is a direct quote.}\cite{RefLabel}
\end{minted}

\section{Adding References}
Your references should be stored in the \texttt{references.bib} file. Each reference should have a unique label that you can use to cite it in your document. Here is an example of a reference entry:
\begin{minted}{bibtex}
@Online{RefLabel,
    author = {Author Name},
    title = {Title of the Article},
    year = {2021},
    url = {https://www.example.com},
    urldate = {2021-01-01}
}
\end{minted}

It is recommended to use a reference manager like JabRef to manage your references.

\section{Including Graphics}
To include graphics, use the \textbackslash includegraphics command from the `graphicx` package and wrap it inside a figure environment. Example:
\begin{minted}{TeX}
\begin{figure}[htp]
    \centering
    \includegraphics[width=0.5\textwidth]{assets/your-image.png}
    \caption{This is a caption.}
    \label{fig:example}
\end{figure}
\end{minted}

\textbf{Note:} If you are using VS Code, you can make use of a snippet to quickly insert these environments by typing \texttt{img}.


\section{Referencing Figures and Listings}
Automatically reference figures, tables, or code listings using \textbackslash autoref. Label your items and reference them as shown:
\begin{minted}{TeX}
\label{fig:example}
As shown in \autoref{fig:example}, this figure demonstrates...
\end{minted}

\section{Glossary Entries and Acronyms}

To add a glossary entry, use the \texttt{\textbackslash newglossaryentry} command in the \texttt{glossary.tex} file. Here is an example:

\begin{minted}{TeX}
\newglossaryentry{api}{
    name={API},
    first={Application Programming Interface (API)},
    plural={APIs},
    description={Application Programming Interface}
}
\end{minted}

To reference the glossary entry in your document, use the \texttt{\textbackslash gls} command. For example:

\begin{minted}{TeX}
The \gls{api} provides a set of functions...
\end{minted}

You can also use the \texttt{\textbackslash glspl} command to make the first use of the term plural. For example:

\begin{minted}{TeX}
\glspl{api} are used to...
\end{minted}

\section{Noting To-Do Items}

To mark a to-do item in your document, use the \texttt{\textbackslash todo} command. For example:

\begin{minted}{TeX}
\todo{Add more details here}
\end{minted}

This will also throw a warning in the document, making it easier to spot and address later. % Remove this line after adding your own content.
% Add more chapters here by adding \input{chapters/chapter_name.tex}
%!TEX root=../main.tex

\chapter{Conclusion}
The conclusion reiterates the key points made and suggests any further research or action that might be necessary.


\printbibliography % Prints the bibliography. This only prints the references that are cited in the document. If no references are cited, this will not print anything.

\end{document}
